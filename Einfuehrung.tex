\chapter{Einführung}
%----------------------------------------------------------
In den folgenden Abschnitten erfolgt eine Einführung in das Themenfeld gedruckte Elektronik. Abschließend wird auf die Zielsetzung der vorliegenden Arbeit „Aufbau und Erprobung einer Schnittstelle zwischen gedruckten Strukturen und Platine“ sowie ihren Aufbau und das Vorgehen eingegangen.
%---------------------------------------------------------
\section{Gedruckte Elektronik}
\label{sec:Gedruckte Elektronik}
Bei {\bf Gedruckte Elektronik} handelt sich um die vollständige bzw. teilweise Herstellung von elektronischen Bauelementen, Baugruppen und Anwendungen mittels Druckverfahren\cite{1}. Mit Hilfe funktionaler Tinten können elektronische Schaltungen auf verschiedene Substraten gedruckt werden. Typischerweise kommen Metall- sowohl Kohlenstoff-basierten leitfähigen Tinten als funktionaler Tinten. Dabei können Substraten wie Glas, Silizium und Poly Folie (PET) eingesetzt werden. Die geringen Kosten und vielfältigen Anwendungen von Paper machen es zu einem attraktiven Substrat, aber seine hohe Rauheit und große Saugfähigkeit machen es für die Elektronik problematisch\cite{2}.

Im Vergleich zur konventionellen Mikroelektronik zeichnet sich die gedruckte Elektronik durch eine einfachere, flexiblere und vor allem kostengünstigere Herstellung aus\cite{3}. Nach \cite{4} hat gedruckte Elektronik einen breiten Anwendungs- und Technologiebereich wie folgenden:

\begin{itemize}
\item Optical codes (1-D bar codes, 2-D bar codes, reactive codes)
\item Hidden or embedded codes (invisible codes, digital watermarks, magnetic codes)
\item RFID tags
\item Electronics (passive components, conductors, circuit boards)
\item Displays (OLED displays, liquid crystal displays, thermochromic displays)
\end{itemize}

Ein weiterer Vorteil der gedruckten Elektronik liegt in ihrer optischen Eigenschaft. Ein Anwendungsbeispiel hierfür ist Indium-zinnoxid (englisch indium tin oxide, {\bf ITO}). {\bf ITO} ist ein optoelektronisches Material, der im sichtbaren Licht weitgehend transparent erscheint. Wegen der optischen Eigenschaften findet ITO in der Industrie breite Anwendung wie Flachbildschirme, intelligente Fenster, polymerbasierte Elektronik, Dünnschichtphotovoltaik und Architekturfenster. ITO wird üblicherweise unter Hochvakuum auf Substrate aufgebracht. Ein dafür häufig eingesetztes Substrat ist Glas, wie in Abbildung~\ref{fig:1.2} dargestellt.

\begin{figure}[H]
\begin{center}
\includegraphics[width=100mm]{bilder/ITO_Glass.jpg}
\end{center}
\caption{ITO-Glass}
\label{fig:1.2}
\end{figure}

Eine grafische Ontologie der gedruckten Elektronik, die die Wechselwirkungen zwischen verschiedenen technischen, wissenschaftlichen und Anwendungsdomänen zeigt, ist in Abbildung~\ref{fig:1.1} dargestellt.

\begin{figure}[H]
\begin{center}
\includegraphics[width=150mm]{bilder/Grafische.png}
\end{center}
\caption{Grafische Ontologie gedruckter Elektroniktechnologien und mögliche Anwendungsfelder}
\label{fig:1.1}
\end{figure}

\section{Zielsetzung}
\label{sec:Zielsetzungs}
Mit Hilfe funktionaler Tinten können Schaltungen aus verschiedenen elektronischen Strukturen und Bauteilen gedruckt werden. Ein dafür häufig eingesetztes Substrat ist Glas. Um die gedruckten Schaltungen zu testen oder in komplexe Schaltungen auf einer Platine zu integrieren, wird eine Schnittstelle zwischen Glassubstrat und Platine benötigt. Diese Schnittstelle dient sowohl der elektrischen Kontaktierung als auch der mechanische Fixierung. Im Rahmen der vorliegenden Arbeit soll eine Schnittstelle aufgebaut werden, die die Kontaktpads auf einem bedruckten Glassubstrat und einer Platine verbindet. Dabei ist die Temperaturempfindlichkeit der gedruckten Elemente zu berücksichtigen. Die Kontaktierung soll in einer sogenannten FlipChip-Anordnung mit Hilfe von Leitklebstoff erfolgen, wie in Abbildung~\ref{FlipChip-Anordnung auf Platine} dargestellt.
\begin{figure}[H]
\begin{center} 
\subfigure[Gedruckte Schaltung]{
\label{Fig.sub.1}
\includegraphics[width=50mm]{bilder/gedruckte_schaltung.png}}
\subfigure[FlipChip-Anordnung]{
\label{Fig.sub.2}
\includegraphics[width=50mm]{bilder/Flipchip_anordnung.png}}
\caption{FlipChip-Anordnung auf Platine}
\label{FlipChip-Anordnung auf Platine}
\end{center}
\end{figure}

\section{Aufbau der Arbeit}
\label{sec:Aufbau der Arbeit}








