\thispagestyle{empty}
\includegraphics[width=0.3\textwidth]{bilder/KITlogo_4c_deutsch_RGB.jpg}
\begin{center}
\huge
\textbf{Aufgabenstellung Masterarbeit}\\
\Large
\vskip 1cm
\textbf{ \textsf{Aufbau und Erprobung einer Schinttstelle zwischen gedruckten Strukturen und Platine} }
\normalsize
\begin{tabular}{ll}
\\
Bearbeiter:   & Zehua Chen\\
Matrikel-Nr.  &2089263\\
Verantwortl. Hochschullehrer  & PD Dr. -Ing. Ingo Sieber\\
Betreuer:    & Dr. -Ing. Liane Koker\\
\end{tabular}
\end{center}

\normalsize
Mit Hilfe funktionaler Tinten können Schaltungen aus verschiedenen elektronischen Strukturen und Bauteilen gedruckt werden. Ein dafür häufig eingesetztes Substrat ist Glas. Um die gedruckten Schaltungen zu testen oder in komplexe Schaltungen auf einer Platine zu integrieren, wird eine Schnittstelle zwischen Glassubstrat und Platine benötigt. Diese Schnittstelle dient sowohl der elektrischen Kontaktierung als auch der mechanischen Fixierung. Im Rahmen der vorliegenden Arbeit soll eine Schnittstelle aufgebaut werden, die die Kontaktpads auf einem bedruckten Glassubstrat und die Kontaktpads einer Platine verbindet. Dabei ist die Temperaturempfindlichkeit der gedruckten Elemente zu berücksichtigen. Die Kontaktierung soll in einer sogenannten FlipChip-Anordnung mit Hilfe von Leitklebstoff erfolgen.
Die Aufgaben im Detail:
\begin{itemize}
\item Einarbeitung in das Themengebiet und die bestehende Hard- und Software
\item Auswahl eines für die Anwendung geeigneten Leitklebstoffs auf Grundlage von Dosierversuchen auf Platinenpads und Glassubstraten
\item Entwicklung eines Handhabungsprozesses für die bedruckten Glassubstrate inklusive Aufbau von Werkstückträgern zur Zuführung sowie Greiferdauswahl
\item Anwendungsspezifische Implementierung der Klebstoffdosierung, der optischen Erkennung der Glassubstrate sowie deren Positionierung in die bestehende Maschinensteuerungssoftware
\item Erprobung des Aufbauverfahrens durch Messung der Leitfähigkeit mit Hilfe von Testsubstraten auf Testplatinen
\end{itemize}

\vskip 1cm
\clearpage