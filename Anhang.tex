\chapter{Anhang}
%\label{sec:Anhang}

\section{Datenblätter}
\label{an:Daten}
{\bf Vakuumchuck}
\begin{figure}[H]
    \centering
    \includegraphics[width=160 mm]{bilder/Zeichnung_Vakuumchuck.pdf}
    \label{an:Vakuumchuck}
\end{figure}
\clearpage

{\bf Kamera}
\begin{figure}[H]
    \centering
    \includegraphics[width=160 mm]{bilder/AnhangKamera.png}
    \label{an:AnhangKamera}
\end{figure}
\clearpage

\includepdf[pages = {1,2}]{KIT-EL325_DE.pdf}\label{pdf:325}

\includepdf[pages = {1,2}]{KIT-EL336_DE.pdf}\label{pdf:336}

\section{Vermessung der verwendeten ITO-Glas und Platine}\label{sec:vermessung}
\clearpage

\section{2D-Zeichnung des Glasträgers}\label{sec:2D}
\clearpage

\section{Quellcode}\label{sec:Quellcode}

\lstset
{
    language = C++,
    frame=leftline, % draw a frame at the top and bottom of the code block
    tabsize=4, % tab space width
    showstringspaces=false, % don't mark spaces in strings
    numbers=left, % display line numbers on the left
    commentstyle=\color{gray}, % comment color
    keywordstyle=\color{red}, % keyword color
    stringstyle=\color{blue}, % string color
    breaklines=true,
    extendedchars=true
}
\begin{lstlisting}
#include<opencv2/opencv.hpp>
#include<opencv2/imgproc.hpp>
#include<opencv2/core.hpp>
#include<opencv2/imgcodecs.hpp>
#include"highgui.h"
#include<iostream>
#include<string>
#include"cv.h"
#include<math.h>
#include<time.h>

#define saveImages 1

//------include files to use the PYLON API-------
#include<pylon\PylonIncludes.h>
#include<pylon\PylonGUI.h>
#include<C:\\Program Files\Basler\pylon 5\Development\Samples\C++\include\ConfigurationEventPrinter.h>

//------include files to use MIMOSE--------------
#include"C:\\Program Files (x86)\VC_API\vcs.h"
#pragma comment (lib, "C:\\Program Files (x86)\\VC_API\\vc_api.lib")

//------Variablen fuer MIMOSE---------------------
short sHandle;
short channel;

using namespace cv;
using namespace std;
using namespace Pylon;
\end{lstlisting}
\clearpage

\section{Daten der Dosierversuche}\label{sec:Dosierversuch}
\clearpage

\section{Anleitungen}\label{sec:anleitungen}
\subsection{Verarbeitung Panacol 325}\label{sec:325}
\clearpage
\subsection{LPKF MotionTools}\label{sec:LPKF}
\clearpage
\subsection{Pylon Viewer}\label{sec:pylon}
\clearpage
\subsection{DIPLOM}\label{sec:diplom}







