\chapter{Grundlagen und Stand der Technik}
\label{sec:Grundlage}
%-------------------------------------------------------
Dieses Kapitel enthält eine Übersicht der wichtigsten Grundlagen der vorliegenden Arbeit. Im Folgenden wird vor allem auf die Klebtechnik eingegangen.
%-------------------------------------------------------
\section{Wichtige Begriffe zur Klebtechnik}
\label{sec:Klebtechnik}
Unter Kleben versteht man das flächige Verbinden gleicher oder verschiedenartiger Werkstoffe unter Verwendung einer meist artfremden Substanz, die an den Oberflächen der zu verbindenden Teile haftet und die Kräfte von einem Fügeteil in das andere überträgt. Einige, für die vorliegende Arbeit wichtige Begriffe, sollen in den folgenden Abschnitten erläutert werden.

\subsection{Adhäsion und Kohäsion}

Mit {\bf Adhäsion} wird das Haften von Stoffen aneinander bezeichnet. Das Adhäsionsvermögen eines Stoffes allein reicht jedoch nicht in der Regel nicht aus, um gute Klebverbindungen herzustellen.

Der innere Zusammenhang eines Stoffes wird als die {\bf Kohäsion} der Verbindung bezeichnet. Spricht man im Zusammenhang mit Klebverbindungen von der Kohäsion, so ist damit immer der Klebstoff gemeint. Übersteigt die Belastung die Kohäsion des Klebstoffes, versagt die Verbindung. Es kommt allerdings nicht nur auf die Festigkeit eines Klebstoffes an, sondern auch auf deren Verformungsfähigkeit. Bei einer Belastung verformen sich die Bauteile. Der anhaftende Klebstoff muss sich ebenfalls mit verformen. Kann er das nicht, weil er zu spröde ist, so kommt es zum Bruch. Spröde Klebstoffe können schon bei geringen Verformung reißen.

Die Benetzung einer Oberfläche kann nur mit viskosen Medien erfolgen. Festkörper können einander nicht benetzen. Viskose Medien haben aber immer eine wesentlich geringere Festigkeit (Kohäsion) als feste Stoffe. Für eine hohe Festigkeit der Klebverbindung wird jedoch eine hohe Kohäsion benötigt, die mit Flüssigkeit nicht erreicht werden kann. Daher wird beim Kleben in der Regel in zwei Schritten vorgegangen. Klebstoffe werden flüssig aufgetragen, somit sie die Oberfläche gut benetzen können. Die Atome der Oberfläche und des Klebstoffes kommen nahe aneinander, Wechselwirkungen entstehen. Es kommt zur Adhäsion. In einem zweiten Schritt wird der flüssige Klebstoff in die feste Phase überführt. 

Adhäsion und Kohäsion werden oft vereinfacht dargestellt, wie Abbildung~\ref{fig:Adhaesion} zeigt. Nach dieser Darstellung findet die Adhäsion an einer Fläche statt. Dies ist jedoch nicht der Fall, denn die Adhäsion bildet sich in einem dreidimensionalen Bereich aus, da es keine ideale ebenen Oberflächen gibt.
\begin{figure}[H]
\begin{center}
\includegraphics[width=90mm]{bilder/Adhaesion.png}
\end{center}
\caption{Vereinfachte Darstellung von Adhäsion und Kohäsion in einer Klebverbindung\\
Adhäsion: Bindungskräfte zwischen Fügeteile und Klebschicht\\
Kohäsion: Bindungskräfte in der Klebschicht}
\label{fig:Adhaesion}
\end{figure}

\subsection{Viskosität}
Die Viskosität eines Klebstoffes hat Einfluss auf seine Anwendung und Verarbeitbarkeit. Die Viskosität eines Klebstoffes ist temperaturabhängig. Zunehmende Temperaturen führen zu sinkenden Viskositäten. Diese gilt es bei der Verarbeitung zu beachten. Wird bei einem mechanisierten Klebstoffauftrag die Klebstoffmenge durch die Öffnungszeit eines Ventils gesteuert, so führen Viskositätsschwankungen zu einem unterschiedlichen Auftrag, da bei hohen Temperaturen mehr durchfließt.

Bei zweikomponentigen Klebstoffen beginnt die Reaktion unmittelbar nach dem Zusammenfügen der beiden Komponenten. Dadurch erhöht sich die Viskosität des Produktes, siehe Abbildung~\ref{fig:Viskositaet}, es wird zunächst zähflüssig und dann fest. Erwärmt sich der Klebstoff aufgrund der Abbindereaktion, so reduziert sich die Klebstoffviskosität zunächst bevor sie ansteigt. Nach Ablauf einer weiteren Zeitspanne wird der so genannte Gelpunkt erreicht. Der Klebstoff beginnt hier, vom flüssige in den festen Zustand überzugehen. Übersteigt die Viskosität ein entsprechendes Maß, bis die Endfestigkeit erreicht wird, so kann der Klebstoff mit der dafür vorgesehenen Technik nicht mehr verarbeitet werden.

\begin{figure}[H]
\begin{center}
\includegraphics[width=80mm]{bilder/Viskositaet.png}
\caption{Zusammenhang zwischen der Viskosität und der Zeit}
\label{fig:Viskositaet}
\end{center}
\end{figure}

\subsection{Topfzeit}
Der Begriff Topfzeit wird beim Verarbeiten von zwei- oder mehrkomponentigen Reaktionsklebstoffen verwendet. Mit Topfzeit wird die Gebrauchs- bzw. Verarbeitungsdauer von Zwei- oder Mehrkomponentenklebstoffen bezeichnet. Die Bezeichnung kommt daher, dass ein Klebstoffe den ,,Topf", in dem er angemischt wurde, bis zum Ende der Topfzeit verlassen haben muss, d.h. er muss innerhalb dieser Zeit verarbeitet sein. Topfzeitangaben sind in den Datenblättern der Klebstoffhersteller zu finden.

DIN 16 920 definiert als Topfzeit für Klebstoffe die ,,...Zeitspanne, in der ein Ansatz eines Reaktionsklebstoffes nach dem Mischen aller Klebstoffbestandteile für eine bestimmte Verwendung brauchbar ist". Diese Definition enthält keine exakt prüfbaren bzw. nach einer Norm überprüfbaren Daten und lässt somit dem Hersteller bei der Festlegung der Topfzeit einen relativ weiten Spielraum. Es kann folglich auch nicht davon ausgegangen werden, dass Topfzeiten unterschiedlicher Hersteller Gleiches aussagen.

Die Temperaturerhöhung verkürzt die Topfzeit. In welchem Maße die Temperatur ansteigt und sich dadurch die Topfzeit verkürzt, hängt von der Reaktivität des Systems und von der Ansatzmengen ab.

\subsection{Abbindezeit}
DIN 16 920 definiert als Abbindezeit für Klebstoffe die ,,...Zeitspanne, innerhalb der die Klebung nach dem Vereinigen eine für die bestimmungsgemäße Beanspruchung erforderliche Festigkeit erreicht". DieseDefinition ist bauteilbezogen und führt folglich nicht zu Klebstoffkennwerten. Somit ist zu beachten, dass die Abbindezeit u.U. nicht anhand von nachmessbaren Eckdaten festgelegt wurde, sondern dass für die Hersteller Spielraum vorhanden ist. Exakte Angaben zur Ermittlung der Abbindezeit und zu den Randbedingungen, unter denen sie ermittelt wurden, sind deshalb hilfreich.

Die Abbindezeit eines Klebstoffes wird von seinem Abbindemechanismus und von den Umgebungsbedingungen, hier vor allen Dingen von der Umgebungstemperatur beeinflusst. Die Festigkeitssteigerung in Abhängigkeit von der Abbindezeit entspricht in der Endphase häufig einer E-Funktion, siehe Abbildung~\ref{fig:Abbindezeit}, d.h. bis zum Erreichen der Endeigenschaften kann eine lange Zeit vergehen. Bei zweikomponentigen Reaktionsklebstoffen, die bei Raumtemperatur abbinden, kann dies bis zu eine Woche und mehr betragen. Grundsätzlich gilt es jedoch zu achten, dass die Funktionsprüfung eines geklebten Teiles erst dann erfolgen kann, wenn der Klebstoff seine Endeigenschaften.

\begin{figure}[H]
\begin{center}
\includegraphics[width=80mm]{bilder/Abbindezeit.png}
\caption{Zusammenhang zwischen der Klebfestigkeit und der Abbindezeit}
\label{fig:Abbindezeit}
\end{center}
\end{figure}


\section{Vor- und Nachteile der Klebtechnik}

Heutzutage wird Klebtechnik zunehmend in der Elektronikindustrie zum Einsatz kommen. Die Möglichkeiten der Klebtechnik und deren Vor- und Nachteile im Vergleich zu anderen Verbindungstechniken sind in Tabelle \ref{tab:Klebtechnik} zusammengefasst.

\begin{table}[H]
    \begin{tabular}{p{7cm}|p{8.2cm}l|l}
    \hline
    {\bf Vorteile} & {\bf Nachteile} \\
    \hline
        keine Wärmebeeinflussung der Fügeteile & begrenzte Warmfestigkeit \\
       \hline
         gleichmäßige Spannungsverteilung & Veränderung der Klebfugen-Eigenschaften bei Langzeiteinsätzen möglich\\
        \hline
         flächige Verbindungen möglich & Reinigung und Oberflächenvorbehandlung der zu verbindenden Teile in vielen Fällen erforderlich\\
         \hline
         unterschiedliche Werkstoffe verbindbar & präzises Einhalten der Fertigungsbedingungen erforderlich\\
         \hline
         Verbinden sehr dünner Fügeteile & oft spezielle Klebvorrichtungen zum Fixieren der Verbindung erorderlich\\
         \hline
         gas- und flüssigkeitsdichtes Fügen, keine Spaltkorrosion & zerstörungsfreie Qualitätsprüfung nur bedingt möglich\\
         \hline
         Verhinderung von Kontaktkorrosion & \\
         \hline
         keine präzisen Passungen der Fügeflächen erforderlich &\\
         \hline
         gute Dämpfungseigenschaften der Verbindung, hohe dynamische Festigkeit &\\
    \hline
    \end{tabular}
    \caption{Eigenschaften von Klebverbindungen}
    \label{tab:Klebtechnik}
\end{table}

\section{Elektrisch leitende Klebstoffe}
\label{sec:Klebstoff}

Mit dem Begriff Klebstoff wird gemäß DIN EN 1692 ein nichtmetallischer Stoff bezeichnet, der Werkstoffe durch Oberflächenhaftung ({\bf Adhäsion}) so verbinden kann, dass die Verbindung eine ausreichende innere Festigkeit ({\bf Kohäsion}) besetzt (Vergleiche Abbildung~\ref{fig:Adhaesion}).

Klebstoffe kommen in der Elektronikindustrie beim Anbringen und Verbinden von Komponenten, beim Energiemanagement, bei Produktsicherheit und Produktschutz sowie bei der Produktidentifikation und bei Abschirmung zum Einsatz. Dabei hat der Klebstoff neben seiner eigentlichen Verbindungsfunktion weitere wichtige Aufgaben zu übernehmen, wie elektrische Leitung, Isolierung, Wärmeleitung, Dichtung, Schutz oder Geräuschdämpfung. Das Beispiel der Verklebung von Bauteilen auf Leiterplatten verdeutlicht diese Funktionalität: Der Klebstoff muss zunächst leicht und schnell zu verarbeiten sein, wobei die Verklebung von Klein- und Kleinstbauteilen oft mittels präzisen Klebstoffdosiergeräten erfolgt. Weiterhin wird gefordert, dass er thermischen Beanspruchungen, insbesondere beim Lötvorgang, widersteht, sowie Unebenheiten in Oberfläche und Belastungen durch die Verbindung unterschiedlicher Materialien ausgleicht.

Elektrisch leitende Klebstoffe erzeugen eine elektrische Verbindung zwischen zwei oder mehreren Kontaktpunkten. Graphitfaser-Vliese und Metall- bzw. metallisierte Partikel sind wesentliche Bestandteile für elektrische leitende Klebstoffe. Gewichtsreduktion, eine sehr flache Bauweise der Endprodukte und eine zuverlässige Elektronik sind nicht die einzigen Vorteile der vielseitig einsetzbaren Klebstoffe. Darüber hinaus erlaubt diese Verbindungsart den Einsatz von kostengünstigen Materialien und Prozessen. Man unterscheidet zwischen isotrop und anisotrop leitfähigen Klebstoffe, siehe Abbildung~\ref{fig:Klebstoffen}.

\begin{figure}[H]
\begin{center} 
\subfigure[Isotrop leitender Klebstoff]{
\label{fig:isotrop}
\includegraphics[width=55mm]{bilder/isotrop.jpg}}
\subfigure[Anisotrop leitender Klebstoff]{
\label{fig:anisotrop}
\includegraphics[width=55mm]{bilder/anisotrop.png}}
\caption{Schematischer Aufbau und Funktionsprinzip von Klebstoffen}
\label{fig:Klebstoffen}
\end{center}
\end{figure}

\subsection{Isotrop und anisotrop leitender Klebstoff}

{\bf Isotrop leitende Klebstoffe} werden dort verwendet, wo der Stromfluss richtungsunabhängig zu erfolgen hat. Diese Klebstoff enthalten z.B. elektrisch leitende Fasern, die sich miteinander verschlingen und so den Stromfluss neben der z-Richtung auch innerhalb der Klebstoffschicht (x- und y-Richtung) ermöglichen, siehe Abbildung~\ref{fig:isotrop}.

Sind dagegen Klebstofffilme erforderlich, die den Stromfluss nur in einer Richtung erlauben, so spricht man von {\bf anisotrop leitende Klebstoffen}, siehe Abbildung~\ref{fig:anisotrop}. Diese nur in z-Richtung leitende Klebstoffe erhält man durch Befüllen des Klebstoffs mit leitfähigen Partikeln. Zwischen den Partikeln wirkt der Klebstoff als Isolator. So finden z.B. warmhärtende Klebstoffe Verwendung, die mit Silber-, Nickel, oder Goldpartikeln gefüllt sein können.

Anisotrop leitende Klebstoffe wurden zunächst vorwiegend für Flüssigkristallanzeigen (Liquid Crystal Displays, LCDs) entwickelt und stellten eine signifikante Erleichterung in deren Fertigung dar. Die sehr feinen Anschlusskontakte der flexiblen Leiterplatten konnten nicht mehr gelötet werden, wodurch der Einsatz von elektrisch leitenden Klebstoffen notwendig wurde. Ein weiteres typisches Einsatzgebiet sind mit Silberpasten bedruckte flexible Polyester-Schaltungen, die an starre Leiterplatten geklebt werden. Klebstoffe sind auch für flexible Schaltungen an flexiblen Folientastaturen geeignet. Speziell hierfür entwickelte Produkte weisen noch weitere Vorzüge auf. Sie lassen sich schnell verarbeiten, weil sie zunächst leicht selbstklebend sind, bevor sie bei 130°C vernetzen. Allerdings benötigen die meisten anisotrop leitende Klebstoffe eine spezielle Kühlung bei Lagerung und Transport unter 0°C, da sie sehr temperaturempfindlich sind.

Je nach Einsatzgebiet gibt es unterschiedliche anisotrop leitende Klebstoffe, die entweder harte oder weiche Partikeln enthalten. Soll eine flexible mit einer flexiblen oder eine flexible mit einer starren Leiterplatten verbunden werden, empfehlen sich Klebstoffe mit harten Partikeln. Sie werden durch Druck in die Leiterbahnen eingebettet und vergrößern dadurch die Kontaktfläche, wodurch gute elektrische Eigenschaften erreicht werden. Zum Anschließen flexibler Schaltungen an Glassubstrate (LCDs) empfehlen sich weiche Partikel, die sich im Kontakt zu harten Oberflächen deformieren und somit die elektrischen Eigenschaften verbessern. Wichtige Parameter bei der Auswahl sind neben dem Klebstofftyp auch Partikeltypgröße und -konzentration. Diese Parameter beeinflussen die elektrischen und mechanischen Eigenschaften, den Klebprozess und die Anwendungsmöglichkeiten. Je mehr Partikel sich in der Matrix befinden, desto besser sind die leitenden Eigenschaften, je weniger, umso besser ist die Klebkraft.

\subsection{Zweikomponenten-Klebstoff}
Zu dieser Gruppe zählen in erster Linie Ployester, kalthärtende Epoxidharze, Polyurethane und auch die Acrylatklebstoffe. Die üblichen Zweikomponenten-Bindemittel müssen vor dem Auftrag in einem bestimmten Verhältnis aus verschiedenen Komponenten angemischt werden. Bei den Acrylatklebstoffen der zweiten Generation ist es möglich, jeweils eine Komponente auf eine der Fügeteiloberflächen aufzutragen, die Teile dann zusammenzufügen und somit die Härtung zu initiieren.

Bei Zweikomponenten-Klebstoffen, die angemischt werden müssen, ist sowohl auf die Einhaltung des vorgeschriebenen Mischungsverhältnisses als auch auf die so genannte Topfzeit, ein von Klebstoffart und Ansatzmenge abhängiger Zeitraum zwischen dem Anmischen und Auftragen, zu achten. Da bereits während der Topfzeit der Vernetzungsprozess langsam beginnt und die Viskosität des Bindemittels sich entsprechend erhöht, fährt ein Überschreiten der Topfzeit zu mangelnder Benetzung der Festkörperoberflächen und demgemäß schlechten Adhäsionseigenschaften in der Klebfuge. Da die Vernetzung von Zwei- oder Mehrkomponenten-Klebstoffen fast immer exotherm erfolgt, erwärmt sich langsamer als große, bei denen das Verhältnis von Volumen zu Wärme abgebender Oberfläche schlechter ist. In diesem Effekt liegt insofern eine besondere Gefahr, als die beginnende Vernetzung des gemischten Klebstoffs oft deswegen nicht anhand der Viskosität zu erkennen ist, weil die Erwärmung gegenläufig zur Vernetzung die Viskosität herabsetzt. Wird ein solches erwärmtes System auf die kalten Fügeteile aufgetragen, verfestigt sich das bereits teilweise vernetzte System sofort und benetzt dementsprechend schlecht.


\section{Auftragsverfahren}
Zum Auftragen von Klebstoffen auf die Klebfläche steht eine große Zahl von Verfahren zur Verfügung. Abbildung~\ref{fig:Auftragsverfahren} zeigt unterschiedliche Möglichkeiten des Klebstoffauftrags auf ebene Flächen.
\begin{figure}[H]
\begin{center}
\includegraphics[width=120mm]{bilder/Auftragsverfahren.png}
\caption{Klebstoffauftrag auf ebenen Flächen mit
unterschiedlichen Auftragsköpfen}
\label{fig:Auftragsverfahren}
\end{center}
\end{figure}
...

\subsection{Auftragverhalten}
Das Auftragsverhalten eines Klebstoffes kann durch seine rheologischen Eigenschaften und sein Ausfließverhalten gesteuert werden. Bild 5.1.6 zeigt Tropfenausbildungen von unterschiedlich eingestellten Klebstoffen. Die Form des Tropfens bzw. der Klebstoffraupe ist bei einigen Anwendungen wichtig. In der Mikroelektronik werden vielfach Klebstofftropfen benötigt, die hoch im Verhältnis zu ihrem Durchmesser sind. Dieses Verhältnis wird nicht nur von der Viskosität des Produktes, sondern auch von den
Ausfließbedingungen beeinflusst, Tabelle 5.1.7. Es ist das Durchmesser (d) – Höhen (h) – Verhältnis sowie die Volumenschwankung bei Verwendung unterschiedlich dicker Auftragsnadeln wiedergegeben. Die Tabelle zeigt weiter, dass unter den vorgegebenen Bedingungen das Auftragsvolumen bei kleinen Volumina sehr stark schwankt. Die Tropfenform kann weiterhin auch durch den Auftragsdruck und die Auftragszeit beeinflusst werden [GLT].

Klebstoffe neigen zum Fadenziehen, d.h. nach dem Auftragen des Klebstoffes kommt es nicht zum spontanen Abriss des Produktes, sondern zwischen dem Klebstoff und der Auftragsdüse bildet sich ein mehr oder weniger langer dünner Faden. Besonders aus dem Bereich der sogenannten Alleskleber ist dieses Problem bekannt. Der Klebstofffaden führt zu einer Verschmutzung der Fügeteile und der Werkzeuge. Im Bereich der Mikroelektronik kann ein Klebstofffaden eine nachfolgende elektrische Kontaktierung unmöglich machen, so dass es zum Ausfall des geklebten Teiles kommt. Abbildung~\ref{fig:Fadenziehen} zeigt das Problem des Fadenziehens von Klebstoffen. Das in a gezeigte Bild stellt einen idealen Fadenabriss dar. Ein kurzer Faden, wie in b, ist noch zulässig, da er sich im Bereich der vorgesehenen Auftragsfläche befindet. Die in c und d gezeigten Fäden können nicht toleriert werden. Sie gehen außerhalb der vorgesehenen Klebfläche auf das Fügeteil nieder und können hier zu Problemen führen.
\begin{figure}[H]
\begin{center}
\includegraphics[width=100mm]{bilder/Fadenziehen.png}
\caption{Problem des Fadenziehens von Klebstoffen}
\label{fig:Fadenziehen}
\end{center}
\end{figure}

\subsection{Klebstoffauftragskontrolle}
Die Kontrolle des aufgetragenen Klebstoffes, sowohl nach der Menge, als auch nach dem Ort des Auftrages, ist vor allem bei einer automatisierten Klebstoffverarbeitung notwendig. Dafür gibt es mehrere Möglichkeiten.
\begin{itemize}
    \item Eine Kontrolle des geförderten Klebstoffes durch Volumenstrommessung. Dies kann mithilfe eines Durchflussmessers gemessen werden. Damit ist allerdings nur feststellbar, dass der Klebstoff in der richtigen Menge gefördert wurde. Leckagen zwischen der Messeinrichtung und der Düse können so jedoch nicht erfasst werden. Weiterhin kann so nicht festgestellt werden, ob der Klebstoff am richtigen Platz aufgetragen wurde.
    \item Kontrolle des aufgetragenen Klebstoffbildes. Hier wird der aufgetragene Klebstoff über eine Kamera erfasst und mit einem vorgegebenen Ist-Bild verglichen, [Brabender]. Der Aufwand ist für diese Kontrolle mit Kamera, Rechner und entsprechender Software recht hoch, doch kann so fast jeder Klebstoffauftrag, erfasst werden.
\end{itemize}

\section{FlipChip-Kontaktierung}
\label{sec:kontaktierverfahren}
...

\clearpage










